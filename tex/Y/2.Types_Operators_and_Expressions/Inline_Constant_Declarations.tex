\subsection{Inline Constant Declarations}

One can use inline constant declarations in order to simplify code

\begin{lstlisting}
    print("Hello!");
\end{lstlisting}
Here, the \texttt{Hello!} is a constant inline string. One can also
use an integer constant like \texttt{1234}, which is by default an
unsigned integer on 32 bits; therefore, the following example is wrong
\begin{lstlisting}
        print_int(u8 : num) : void {
            print("%d\n", num);
        }

        [...]

        main(u32 : ac, ubyte[]* : av) : u32 {
            print_int(23);
        }
\end{lstlisting}
Instead, one must use
\begin{lstlisting}
        print_int(u32_to_u8(23));
\end{lstlisting}
Which is valid by the type conversions standard of the Y language, see
\ref{Type_conv} section for more information.

In order to gain some performance at run-time, by avoiding a function call,
one can use constants cast
\begin{lstlisting}
        print_int(__u8(23));
\end{lstlisting}
It will change the internal type of the variable at compilation time,
acting like a cast. See the \ref{comp_cast} section for more information.

The value of an integer can be specified in octal or hexadecimal
instead of decimal. A leading \texttt{0} (zero) on an integer
constant means octal; a leading \texttt{0x} or \texttt{0X} means
hexadecimal.  For example, decimal 31 can be written as
\texttt{037} in octal and \texttt{0x1f} or \texttt{0x1F} in hex.

A character constant is an integer, written as one character within
single quotes, such as \texttt{'x'}. The value of a character constant
is the numeric value of the characters in the machine's character set.
For example, in the ASCII character set the character constant \texttt{'0'}
has the value \texttt{48}, which is unrelated to the numeric value \texttt{0}.

Certain characters can be represented in character and string constants
by escape sequences like \texttt{\textbackslash n} (newline); these sequences look
like two characters, but represent only one. In addition, an arbitrary
byte-sized (ubyte) pattern can be specified by
\begin{lstlisting}
        '\ooo'
\end{lstlisting}
Where \texttt{ooo} is one to three octal digits (0..7) or by
\begin{lstlisting}
        '\xhh'
\end{lstlisting}
where \texttt{hh} is one or more hexadecimal digits (0..9, a..f, A..F).
The complete set of escape sequences is
\\\\
$\begin{array}{|l|l|}
\hline
Sequence & Value\\
\hline
\backslash a & \text{alert (bell) character} \\
\hline
\backslash b & \text{backspace} \\
\hline
\backslash f & \text{formfeed} \\
\hline
\backslash n & \text{newline} \\
\hline
\backslash r & \text{carriage return} \\
\hline
\backslash t & \text{horizontal tab} \\
\hline
\backslash v & \text{vertical tab} \\
\hline
\backslash \backslash & \text{backslash} \\
\hline
\backslash ? & \text{question mark} \\
\hline
\backslash ' & \text{single quote} \\
\hline
\backslash " & \text{double quote} \\
\hline
\backslash ooo & \text{octal number} \\
\hline
\backslash xhh & \text{hexadecimal number} \\
\hline
\end{array}$
\\\\
The character constant \texttt{'\textbackslash 0'} represents the character
with value zero, the null character.

A \texttt{constant expression} is an expression that involves only constants.
Such expressions may be evaluated at during compilation rather than run-time,
and accordingly may be used in any place that a constant occur, as in
\begin{lstlisting}
    ubyte[100 + 20]     str;
\end{lstlisting}
will be compiled as
\begin{lstlisting}
    ubyte[120]     str;
\end{lstlisting}

A string constant, or string literal, is sequence of zero or more
characters surrounded by doubles quotes, as in
\begin{lstlisting}
    "I am a string"
\end{lstlisting}
or
\begin{lstlisting}
    "" /* The empty string */
\end{lstlisting}
The quotes are not part of the string, but serve only to delimit it. The
same escape sequences used in character constants apply in strings;
\textbackslash " represents the double-quote character. String constants
can be concatenated at compile time:
\begin{lstlisting}
    "hello, " "world"
\end{lstlisting}
is equivalent to
\begin{lstlisting}
    "hello, world"
\end{lstlisting}
