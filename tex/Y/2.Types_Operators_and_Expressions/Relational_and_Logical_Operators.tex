\subsection{Relational and Logical Operators}

The relational operators are

$\begin{array}{llll}
> & >= & < & <=
\end{array}$

They all have the same precedence. Just below them in precedence are
the equality operators:

$\begin{array}{ll}
== & !=
\end{array}$

Relational operators have lower precedence than arithmetic operators,
so an expression like \texttt{i < lim - 1} is taken as \texttt{i < (lim - 1)}
, as would be expected.

The logical operators are

$\begin{array}{ll}
\&\& & ||
\end{array}$

Expressions connected by \texttt{\&\&} or \texttt{||} are evaluated
left to right, and evaluation stops as soon as the truth or falsehood
of the result is known. By definition, the numeric value of a relational
or logical expression is 1 if the relation is true, and 0 if the relation
is false.
